\chapter{Geschäftsmodell und Organisation}

Geschäftsplan und Organisation Stichpunkte:
sehr kurze Beschreibung Geschäftsmodell:
Die (weltweiten) User können sich die Fotoapp kostenlos herunterladen und sind angehalten pro Login-Session mindestens ein
besonderes Foto (kostenfrei) hochzuladen. Somit werden die besonderen Fotoinhalte, welche letzten Endes weitere User anlocken sollen, von der App-Community kostenlos frei gesteuert,
ohne das wir für diese wertvollen Inhalte bezahlen. Je mehr likes einzelne Fotos erhalten, umso mehr Punkte erhält der Hochlader.
Diese Punkte kann er für sonst zahlungspflichtige Features eintauschen oder auch als Eigenmarketing seinen Kanal benutzen. Andere User können besondere Inhalte/Features gegen Punkte/Geld freikaufen,
wodurch das Unternehmen sein Geld verdient.

Kernaktivität: Bereitstellung einer Sharing-Plattform auf der ganz besondere Foto-Inhalte von privaten Usern veröffentlicht werden können.


Das Produkt (Fotoapp) wird hochskaliert und weltweit angeboten. Die interessanten Fotoinhalte werden von den User mindestens bei jedem
Login in die App erstellt, auch weitere Fotos können geladen werden -> skaliert sehr hoch. 
•Ressourcen: Beschreiben Sie, welche Ressourcen notwendig sind, damit Ihr Geschäftsmodell funktioniert. 
Diese Ressourcen können folgender Art sein: physisch (z. B. Gebäude, Maschinen, Fahrzeuge), geistiges Eigentum
 (z. B. Patente, Marken, Software, Kundenprofile) oder finanziell (z. B. Cash, Kreditlinie, Investment) sowie 
 Kompetenzen und „Know-how“ beispielsweise vom Gründerteam und/oder Personal mit spezifischen Qualifikationen.
 
 xxxxxxxxx
 notwendige Ressourcen:
 -physisch:(z. B. Gebäude, Maschinen, Fahrzeuge)
 - geistiges eigentum:z. B. Patente, Marken, Software, Kundenprofile)
  nicht patentierbar, keine besonderen algorithmen, ausgang eu, nicht usa
 -finanziell (z. B. Cash, Kreditlinie, Investment)
 -Kompetenzen und „Know-how“ beispielsweise vom Gründerteam und/oder Personal mit spezifischen Qualifikationen:
 Ressourcen: 
 vorerst stellen die vier Mitglieder des Teams alle benötigten technischen Ressourcen, da alle Informatiker sind:
 Die Kompetenzen erstrecken sich über die eingesetzte Programmiersprache und Frameworks, Deployment durch Studium und praktische Erfahrung (PSS),
 die Testkompetenzen ebenfalls, Dokumentation der Software und Entwicklungsorganisation (Scrum)
 Zusätzlich ist im Team Teammitglieder Startup-Erfahrung und Pitch-Erfahrung, vorhanden, sodass man über die ein oder anderen Stolpersteine bereits bekannt sind, genauso
 wie die Wichtigkeit des Mentorings, für das die Investoren natürlich gezielt ausgesucht werden sollen (nicht nur finanzielle Hilfe, sondern auch Domänen-Kompetenz und Hilfe bei Markteintrittsbarrieren
 und professionelle Unternehmensentwicklung und Führung)
 Was fehlt, wäre das Geld für Marketing,es soll versucht werden, dieses einzuholen
 -bargeldbestände können aktuell beschränkt werden auf knapp 500-1000€ für Reise und Unterkunft zu Investoren/Messen (initialkosten für investorenanwerbung)
 -weiterbildungskosten (derzeit abgedeckt), spätere weiterbildung nicht auszuschliessen (wirtschaftlich/technisch)
 -materialkosten für werbematerial auf messen oder für investoren
 
 
 
 Partnerschaften:
 Partnerschaften wird es zuerst mit Investoren zwangsläufig geben.
 Erwartungen an diese: Startup Kompetenz im Medienbereich/Werbebereich.
 Kompetenzen in Markteinführung, Weiterenticklung, marketing und Vertrieb.
 
 technische Partnerschaft: Server von Amazon. 
 Eigenschaften: gute flexible Kostenstruktur, zuverlässige in Europa platzierte 
 Server für besseren Datenschutz.Hierfü werden natürlich Kosten fällig.
 
 Partnerschaften können sich evtl aus den Vertriebswegen ergeben, dass Firmen
 gegen kleinen Anteil zu festgelegten Konditionen /festgelegte Art unser Unternehmen vermarkten
 bzw. aktiv nutzen und somit ihre eigene Community auf unsere App aufmerksam machen.
 
 Möglicherweise Partnerschaften mit besonderen Erlebnisunternehmen (Otto-200E Erlebnisgutscheine), Reiseuntrnehmen,
 mit denen ergeben sich synergien. Das ganze ist allerdings nur ein Ausblick.
 Es wird nicht nur werbung für unsere app gemacht, die ja besondere momente festhält, sondern auch für die unternehmen selbst,
 da diese besondere momente bewerben.
 
 z.b. mit ausrüsterfilmen: wie z.b. canon spiegelreflexkameras oder samsung-hersteller mit ganz besonders guter kamera, die geeignet istu
 um besondere momente festzuhalten auf unserer app
 solche Unternehmen haben ganz andere kernaktivitäten, die sich mit unseren perfekt ergenzen.
 somit keine konkurrierende Interessenverfolgung, denn diese unternehmen konkurrieren z.b.
 auch nicht mit unseren konkurrenten (Instagram, pinterest, etc.)
 
 -make or buy:
 für die kernaktivitäten bestehen derzeit alle benötigten technischen kompetenzen im eigenen team.
 -für das marketing könnte eine marketing-firma engagiert werden, bzw. vertriebsleute eingestellt werden,
 da professioneller marketing/vertrieb nicht zu unseren kernkompetenzen gehört.
 -später supportdienst outsourcen

 kostenstruktur: wichtigste kostentreiber:
 marketing/vertrieb
 die meisten leute haben gute smartphones mit top-ausgestatteten kameras, daher ist das keine sorge
 
 -value proposition/nutzenversprechen: wir ermöglichen das sehen und teilen von ganz besonderen momenten im fotoformat.
 
 -kundensegment: erstlinig die junge generation: ab 16 bis 35, aber natürlich auszuweiten für alle alterklassen, wobei
 die älteren natürlich nicht so häufig auf smartphones/social networks herumtreiben.
 aus diesem grund wird auch gezielte werbung für diese Zielgruppe gemacht, siehe marketing.
 
 -Vertriebs/Kommunikationskanäle:
 Über facebook zielgruppe wählen, dann bezahlen, rest übernimmt facebook.
 Kriterien für facebook-wahle zum einstellen:
 Alter: 16-34
 -hobbys/interessen: reisen
 -hauptsächlich europa, Nordamerika asien  zuerst in dieser reihenfolge.
 damit sollte schon genug marketing geld fürs erste ausgebucht sein.
 
 erlösstruktur:Folie 172
 xxxxxxxxxxxxxxxxxxxxxxxxxxxxxx
